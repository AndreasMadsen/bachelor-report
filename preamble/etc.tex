%!TEX root = ../Thesis.tex

\usepackage{mathtools}	% Det meste matematik (indeholder ams­math og rettelser)
\usepackage{xfrac}		% Flere fracs (\sfrac{}{})
\usepackage{listings}	% Indsæt code
\usepackage{todonotes}	% Cool todo notes, [disable] skjuler todos
\usepackage[backend=bibtex,bibstyle=ieee,citestyle=numeric-comp]{biblatex} % Benyt BibLaTeX til formatering

\usepackage{subcaption}	% Tillader subfigure, subtable samt \captions
\usepackage{mathtools}	% Det meste matematik (indeholder ams­math og rettelser)
\usepackage{xfrac}		% Flere fracs (\sfrac{}{})
\usepackage{listings}	% Indsæt code
\usepackage{todonotes}	% Cool todo notes, [disable] skjuler todos

%listing settings, æøå support, font config, line number, left lines
\lstset{
    breakatwhitespace=false, breaklines=true,
    inputencoding=utf8, extendedchars=true,
    literate={å}{{\aa}}1 {æ}{{\ae}}1 {ø}{{\o}}1 {Å}{{\AA}}1 {Æ}{{\AE}}1 {Ø}{{\O}}1,
    keepspaces=true, showstringspaces=false, basicstyle=\small\ttfamily,
    frame=L, numbers=left, numberstyle=\scriptsize\color{gray},
    keywordstyle=\color{SteelBlue}\ttfamily,
    stringstyle=\color{IndianRed}\ttfamily,
    commentstyle=\color{Teal}\ttfamily,
} 

\DeclareGraphicsExtensions{.pdf,.eps,.png,.jpg,.gif}	% ændre til .png, .jpg for hurtig visning

\newcommand\defeq{\mathrel{\overset{\makebox[0pt]{\mbox{\tiny def}}}{=}}}
